%-------------------------------------------------------------------------------
%	SECTION TITLE
%-------------------------------------------------------------------------------
\cvsection{University Projects}


%-------------------------------------------------------------------------------
%	CONTENT
%-------------------------------------------------------------------------------
\begin{cventries}

%========================== ACC ==========================
%---------------------------------------------------------
  \cventry
    {Carnegie Mellon University} % Organisation
    {AWS Auto-Scaling Design} % Project
    {Pittsburgh, PA} % Location
    {Mar 2023} % Date(s)
    {
      \begin{cvitems} % Description(s) of project
        \item {Designed an AWS auto-scaling controller that automatically collects workload metrics 
        and adjust the number of running EC2 instances. The controller increases aggregate request 
        processing speed by twice, while decreases the aggregate EC2 cost by 7\%.}
      \end{cvitems}
    }

%---------------------------------------------------------
  \cventry
    {Carnegie Mellon University} % Organisation
    {Apache Spark Interative Machine Learning Training} % Project
    {Pittsburgh, PA} % Location
    {Apr 2023} % Date(s)
    {
      \begin{cvitems} % Description(s) of project
        \item {Developed a distributed logistic regression algorithm (Python) in Apache Spark that leverages 
        sparse linear algebra and join-based communication between distributed dataset. Deployed on a cluster 
        of 16 nodes and trained excessively huge models within reasonable time limits.}
      \end{cvitems}
    }

%---------------------------------------------------------
  \cventry
    {Carnegie Mellon University} % Organisation
    {Intelligent Job Scheduler in Kubernates} % Project
    {Pittsburgh, PA} % Location
    {May 2023} % Date(s)
    {
      \begin{cvitems} % Description(s) of project
        \item {Implemented a Kubernates job scheduler (Go) using Kube Batch for randomly arriving jobs with 
        different scheduling preferences. Deployed on a multi-rack data cluster consisting of heterogeneous 
        hardwares, and achieved more than 3 times improvements compared to default scheduler in terms of average
        job completion time.}
      \end{cvitems}
    }

%========================== 17514 ==========================
%---------------------------------------------------------
  \cventry
    {Carnegie Mellon University} % Organisation
    {Santorini Game Development} % Project
    {Pittsburgh, PA} % Location
    {Nov 2022} % Date(s)
    {
      \begin{cvitems} % Description(s) of project
        \item {Developed a board game called Santorini, using Java for backend, and Typescript \& ReactJS for 
        frontend.}
        \item {Designed HTTP-based RESTful APIs between backend and frontend to achieve low-coupling and 
        high cohesion of the entire program.}
        \item {Deviced flexible framework interface in backend that allows users to add their own card plugins 
        by implementing just a few lines of codes.}
      \end{cvitems}
    }

%---------------------------------------------------------
  \cventry
    {Carnegie Mellon University} % Organisation
    {Parallel Four-color Map Solver} % Project
    {Pittsburgh, PA} % Location
    {May 2023} % Date(s)
    {
      \begin{cvitems} % Description(s) of project
        \item {Innovated and implemented parallel breadth-first-search algorithm (C++) using OpenMP that minimizes 
        inter-node dependencies and cache false-sharing. Achieved super-linear scaling (more than 10x speedup) on 
        a cluster of 8 nodes.}
        \item {Constructed an interactive web client interface (JavaScript) using P5.js that allows users to draw 
        arbitrary maps and play around with the parallel map solver at the backend cluster. Built a server (C++) 
        on the backend cluster that receives requests from and provide solutions to web client through HTTP.}
      \end{cvitems}
    }

\end{cventries}
